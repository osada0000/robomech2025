\section{伸縮するブラキエーションロボットの製作}
\subsection{全体}
% コンセプトを基に実機を製作した.製作したロボットのCAD図,写真をそれぞれ図\ref{4.2},図\ref{tel_act}に示す.ロボットの重さは$2.5$ kgであり,最も縮めた場合の幅・奥行・高さは$150$×$80$×$500$ mm,最も伸ばした場合の幅・奥行・高さは$150$×$80$×$740$ mmである.腕を鉛直方向にまっすぐ伸ばした形であるが,正面方向からはできる限り上下対称になるように設計した. 
% \begin{figure}[tb]
%         \begin{center}
%           \includegraphics[keepaspectratio, width=0.8\linewidth]{/home/hijiri/template_robomech2024_tex/fig/4.2.eps}
%           \caption{CAD diagram of the telescoping brachiation robot}
%           \label{4.2}
%         \end{center}
% \end{figure}
% \begin{figure}[tb]
%         \begin{center}
%           \includegraphics[keepaspectratio, width=0.6\linewidth]{/home/hijiri/template_robomech2024_tex/fig/telescopic _actual.eps}
%           \caption{Overall view of telescoping brachiation robot}
%           \label{tel_act}
%         \end{center}
% \end{figure}
% \subsection{伸縮機構部}
% 伸縮機構部のCAD図を図\ref{4.5}に示す.機構にはラックピニオン機構を採用し,中心部のブラシレスDCモータに取り付けられた歯車によって左右のラックが上下,ロボット全体も伸縮する仕組みになっている.また,左右方向の抑えとしてリニアブッシュを使用した.これにより,リニアガイドなどと比較して軽量な機構でスムーズな伸縮ができる.ブラシレスDCモータにはMAXONのEC22 40Wにギア比128のギアヘッドを取り付けたものを使用した.
% \begin{figure}[tb]
%         \begin{center}
%           \includegraphics[keepaspectratio, width=\linewidth]{/home/hijiri/template_robomech2024_tex/fig/4.5.eps}
%           \caption{Expansion and contraction unit}
%           \label{4.5}
%         \end{center}
% \end{figure}
% \subsection{グリッパ部}
% グリッパ部のCAD図を図\ref{4.3}に示す.POMで製作した爪がサーボモジュールで駆動し,グリッパ全体がウォームギヤ機構によってヨー軸に回転することでロボットの進行方向を変えることができる.サーボモジュールには双葉電子工業のRS406CB,ウォームギヤ機構のDCモータにはMAXONのRE16 4.5Wにギア比84のギヤヘッドを付けたものを使用した.また,爪は図\ref{4.4}に示すようにフォームクロージャ型に設計し,トルクをかけずにバーにぶら下がることができる.
% \begin{figure}[tb]
%         \begin{tabular}{ccc}
%             \begin{minipage}{.6\textwidth}
%                 \centering
%                 \includegraphics[width=0.8\linewidth]{/home/hijiri/template_robomech2024_tex/fig/4.3.eps}
%                 \caption{Gripper unit}
%                 \label{4.3}
%             \end{minipage}
%             \begin{minipage}{.4\textwidth}
%                 \centering
%                 \includegraphics[width=0.8\linewidth]{/home/hijiri/template_robomech2024_tex/fig/4.4.eps}
%                 \caption{Clow design}
%                 \label{4.4}
%             \end{minipage}
%         \end{tabular}
%     \end{figure}
% \subsection{電装部}
% 回路の構成図を図\ref{4.6}に示す.ロボット全体の制御にはRaspberry Pi4 ModelBとArduino Dueを使用した.グリッパ部のDCモータはモータドライバを介してArduino Dueに接続されている.また,伸縮機構部のブラシレスDCモータはEPOS2 24/5を介してArduino Dueに接続されており,角度,角速度,トルク制御を行う.また,エンコーダ部にはエンコーダカウンタ(MikroElektronika COUNTER CLICK)が接続されている.ロボット全体の姿勢取得にはIMU(Adafruit BNO055)を使用した.
% \begin{figure}[tb]
%         \begin{center}
%           \includegraphics[keepaspectratio, width=\linewidth]{/home/hijiri/template_robomech2024_tex/fig/4.6.eps}
%           \caption{Circuit diagram}
%           \label{4.6}
%         \end{center}
% \end{figure}