\section{伸縮による振動拡大}
\subsection{伸縮による振動拡大のメカニズム}
% 図\ref{3.1}に棒の上を質点が移動する振り子のモデルを示す.これは棒状ブラキエーションロボットを簡略化したものである.ここで,振り子の半径方向と鉛直下向き線がなす角を$\phi$[rad],振り子の半径方向における回転中心からの質点の位置を$r$[m],振り子の質量を$\rm{m_R}$[kg],振り子の慣性モーメントを$\rm{I_R[kgm^2]}$,質点の質量を$\rm{m_M}$[kg],振り子の長さをl[m]とする.
% \begin{figure}[tb]
%         \begin{minipage}[b]{0.4\linewidth}
%           \centering
%           \includegraphics[keepaspectratio, width=\linewidth]{/home/hijiri/mthesis-template-utf8/fig/3.1.eps}
%           \caption{Simplified model}
%           \label{3.1}
%         \end{minipage}
%         \begin{minipage}[b]{0.55\linewidth}
%           \centering
%           \includegraphics[keepaspectratio, width=\linewidth]{/home/hijiri/mthesis-template-utf8/fig/excite_flow.eps}
%           \caption{Excitation flow}
%           \label{3.2}
%         \end{minipage}
%       \end{figure}
% \begin{figure}[tb]
% \begin{center}
% \includegraphics[keepaspectratio, width=0.3\linewidth]{/home/hijiri/mthesis-template-utf8/fig/3.1.eps}
% \caption{Simplified model}
% \label{3.1}
% \end{center}
% \end{figure}
この系の力学的エネルギーは
\begin{eqnarray}
\begin{split} 
\equlabel{3.1}
T &= m_R \frac{l^2}{8} \dot{\phi}^2 + m_M \left(\frac{l}{2}-r\right)^2 \!\frac{\dot{\phi}^2}{2} \\
&\quad + m_M \frac{\dot{r}^2}{2} + I_R \frac{\dot{\phi}^2}{2},
\end{split}
\end{eqnarray}
位置エネルギーは
\begin{eqnarray}
\equlabel{3.2}
U = - g \, m_R \frac{l}{2} \cos(\phi) - g \, m_M \left( \frac{l}{2}-r \right) \cos(\phi)
\end{eqnarray}
で表わされる.

ここで,質点の位置$r$が
\begin{eqnarray}\label{cond_1}
        r
        =
        \begin{cases}
        r_\text{MAX} &\mathrm{if}\,\mathrm\phi \dot{\phi} > 0 \lor \mathrm\phi = 0\\
        r_\text{MIN} &\mathrm{if}\,\mathrm\phi \dot{\phi} < 0 \lor \dot{\phi} = 0
        \label{eq:setpoint}
        \end{cases}
\end{eqnarray}
に従って移動する場合(図\ref{3.2}に示す重心の移動),
$\left|\theta\right|<=\pi$では,系の力学的エネルギは,半周期ごとに次のように増加する.
\begin{eqnarray}\label{M_E_excite}
\begin{split} 
\Delta E_{t_{0}}^{t_{2}}&=\Delta T_{t_{0}}^{t_{0}^{+}}\\
&+g m_{M}\left(r_{\mathrm{MAX}}-r_{\mathrm{MIN}}\right)\left(1-\cos \left(\phi\left(t_{1}\right)\right)\right)
\end{split}
\end{eqnarray}
ここで,$\Delta T_{t}^{t^{+}}$は
\begin{eqnarray}
\Delta T_{t}^{t^{+}} & =\left(\frac{M_{11}\left(r\right)}{M_{11}\left(r^{+}\right)}-1\right) \frac{1}{2} M_{11}\left(r\right) {\dot{\phi}}^{2}, \\
M_{11}(r) & = m_{R} \frac{l^{2}}{4}+m_{M}\left(\frac{l}{2}-r(t)\right)^{2}+I_{R}
\end{eqnarray}
である.
式\ref{M_E_excite}の右辺第$1$項は常に正であり,第$2$項に関しても$\left|x_{1}\right|<=\frac{\pi}{2}$では正になるため,励振できる.
しかし,これを実機へ適応させる際に,瞬間的に質点を移動させることは現実には不可能なため,式\ref{cond_1}をより一般化した条件である
\begin{eqnarray}
r
=
\begin{cases}
r_\text{MAX} &\mathrm{if}\,\mathrm\phi \dot{\phi} > 0,\\
r_\text{MIN} &\mathrm{if}\,\mathrm\phi \dot{\phi} < 0,\\
0 &\mathrm{if}\,\mathrm\phi \dot{\phi} = 0.
\label{eq:setpoint}
\end{cases}
\end{eqnarray}
を満たすように重りの位置をフィードバック制御する.
ここで,$\phi$はロボットの角度,$\dot{\phi}$はロボットの角速度である.
\subsection{実機によるセンサフィードバックな振動拡大}
前節で製作した実機で振動拡大実験を行った.式\ref{eq:setpoint}を基に目標ロボット長さを計算し,制御量をロボット長さ,操作量をモータトルクとしてPD制御した.目標ロボット長さは,IMUで取得した角度と角速度の積の符号によって切り替え,$620$ mmもしくは$770$ mmとした.また,実験環境は図\ref{3.5}と同じ環境を使用した.
時間とロボット角度$\theta$,ロボット角速度$\dot{\theta
}$,目標ロボット長さ$r_{\rm{m_{target}}}$,実ロボット長さ$r_{\rm{m}}$の関係を図\ref{4.7}に示す.
結果として,センサフィードバックにより伸縮するブラキエーションロボットにおいても,ロボットの角度を増幅させることができた.
% \begin{figure}[tb]
%         \begin{center}
%           \includegraphics[keepaspectratio, width=\linewidth]{/home/hijiri/template_robomech2024_tex/fig/excite_2.eps}
%           \caption{Relationship between time and each parameters}
%           \label{4.7}
%         \end{center}
% \end{figure}
