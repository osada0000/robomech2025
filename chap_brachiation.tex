\section{伸縮を利用したブラキエーション動作}
ブラキエーション動作の途中で,ロボットが伸びることにより空中相を含まずに遠くのバーへ移動することができる。
これを実験により実証した.
実験におけるロボットの流れは,
\begin{enumerate}
  \item 全長が$650$ mmの状態で,両グリッパでバーを掴んだ状態から片グリッパをバーから離す
  \item バーを回転軸として,最下点まで揺れる
  \item 最下点において100 mm縮み,振動を拡大
  \item 最上点付近で全長790 mmまで伸び,次のバーを掴む
\end{enumerate}
グリッパを掴む・離す,伸縮するタイミングは試行錯誤して決定した.
また,動作の途中で,ロボットの全長を縮めて重心移動し,振動を拡大する.
これは,空気抵抗やケーブルなどの影響によって,励振無しでは次のバーへ到達できないためである.
伸縮するブラキエーションの様子を図\ref{b_1}に示す.
% \begin{figure}[tb]
%   \begin{center}
%     \includegraphics[keepaspectratio, width=\linewidth]{/home/hijiri/template_robomech2024_tex/fig/brachiation_exp.eps}
%     \caption{Brachiation motion using expansion and contraction}
%     \label{b_1}
%   \end{center}
\end{figure}
結果として,140 mm先のバーへ到達することができた.
空中相と呼ばれる,飛ぶようなフェーズを含むことで遠くのバーへ到達可能なブラキエーションの研究はあるが,
それに対し,伸縮を利用することで遠くのバーへ到達できることを示した.