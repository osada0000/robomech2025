\documentclass{jarticle}
\usepackage{robomech}
\usepackage{graphicx}
\usepackage{ikuo}%%便利コマンド集.
% \usepackage{siunitx}
\usepackage{jtygm}
\usepackage{bm}
\usepackage{balance}
\usepackage{amssymb}
% \usepackage{here}
\usepackage{amsmath}
\usepackage{url}
\usepackage[hang,small,bf]{caption}
\usepackage[subrefformat=parens]{subcaption}
% \usepackage{graphics} % for pdf, bitmapped graphics files
% \usepackage{epsfig} % for postscript graphics files
\newcommand{\FIGDIR}{./fig}
\newcommand{\ctext}[1]{\raise0.1ex\hbox{\textcircled{\scriptsize{#1}}}}
\setcounter{topnumber}{4} 

\begin{document}
\makeatletter
\title{立ち上がり時支援とエネルギー消費なしの中腰支援が可能な\\空気圧回生型ウェアラブル椅子}
{\vspace{-7 mm} }
{Pneumatic Regenerative Wearable Chair for \\
Supporting Up and a Half-sitting Posture without Energy Consumption}
{}

\author{
\begin{tabular}{ll}
    \hspace{1zw}  長田京右(東京農工大)& 
    \hspace{1zw}森下克幸(東京農工大)\\
    ○正\hspace{1zw}水内郁夫(東京農工大)\\
 % ※協賛・後援団体の会員資格で発表される場合は「正・学」は不要です。
 \end{tabular}
 % &\\
 \vspace{1zh} \\
 \begin{tabular}{l}
{\small Kyosuke OSADA, TUAT, osada-rm25@mizuuchi.lab.tuat.ac.jp}\\
{\small Katsuyuki MORISHITA, TUAT, morishita@mizuuchi.lab.tuat.ac.jp}\\
{\small Ikuo MIZUUCHI, TUAT, ikuo@mizuuchi.lab.tuat.ac.jp}
\end{tabular}
}
\makeatother

\abstract{ \small 
Prolonged sitting and standing up at work is physically demanding.
The physical burden of long hours of sitting and standing up is great.
Assistive devices have been studied to reduce the physical burden.
The work that requires assistive devices is often performed outdoors, 
and it is difficult to obtain a power source. Therefore, long-term battery 
power operation is required. This study aims to develop a wearable pneumatic 
device that can support standing up and a half-sitting posture. Incorporating a 
pneumatic regeneration mechanism can reduce the physical burden and realize 
long-term operation.  We fabricated wearable chair to provide mid-back and 
stand-up support. In addition, the effectiveness of the pneumatic regeneration 
mechanism was verified through experiments.
}

\date{} % 日付を出力しない
\keywords{Long time operation, Air regeneration, Wearable chair}

\maketitle
\thispagestyle{empty}
\pagestyle{empty}

\small
% \section{緒言}
作業における長時間の中腰姿勢作業や立ち上がり動作は身体的負担が
大きい.身体的負担を軽減するためにアシスト装置の研究が行われている.
これらの作業は野外での場合が多く,電源を得ることが困難であるため,
バッテリーでの長時間動作が求められる.
中腰姿勢作業の負担軽減を行うために様々なウェアラブル椅子が研究されている\cite{noonee}\cite{yao}\cite{osada1}\cite{isu}.
これらは,中腰支援維持をすることが
可能であるが,立ち上がり支援を行うことができない.Magdumらの研究では,
ウェアラブルな椅子の高さ調整に空気圧シリンダを利用することで,中腰支援と
立ち上がりが可能である.空気圧シリンダに圧縮空気を印加し,弁を閉じることで
中腰姿勢維持をエネルギー消費なしに中腰姿勢を維持することが可能である\cite{nita}.
しかし,使用した圧縮空気を大気圧解放するためエネルギーが無駄になっている.
このエネルギーの無駄を省くため,一度使用した圧縮空気を再利用する空気圧
回生機構が存在する\cite{kumakura}\cite{kuma}\cite{sasaki}.\\
 本研究では,立ち上がり時支援と中腰支援が可能な空気を動力源
とするウェアラブル椅子を製作し,身体的負担を軽減するとともに,空気圧回生機構を
組み込むことにより長時間動作を実現する.
% noonee社のChairless Chair 2.0\cite{noonee}や
% アルケリスは\cite{osada1},ウェアラブルな椅子であり,
% \section{立ち上がり時支援と中腰支援が可能な装置の提案}
立ち上がり時支援と中腰支援が可能なウェアラブル椅子の模式図を
\figref{concept.eps}に示す.
圧縮空気を利用して伸縮する空気圧シリンダにより立ち上がり時支援と中腰支援を行う.
空気圧シリンダに圧縮空気を印加し,弁を閉じることで中腰姿勢維持をエネルギー
消費なしに中腰姿勢を維持するができ,長時間使用に貢献することができる.
\subsection{小型ウェアラブル椅子の機構および製作}
試作した小型のウェアラブル椅子を\figref{equipment}の(a)に示す.
シリンダ非伸長時の全長は270 mm,重さは0.56 kgである.使用するシリンダの
ストロークが60 mmである.
\figref{equipment}の(b)に関節構造を取り入れることでウェアラブル椅子移動
の妨げにならないような設計となっている.
スピンロック(イマオコーポレーション製)のノブを90 deg回すことにより座面
のロックを行うことが可能である.
\subsection{小型ウェアラブル椅子の動作試験}
製作した小型ウェアラブル椅子の上に重量3.1 kgのROBOTIS社製の
ヒューマノイドロボットDARwin-OPを
乗せて中腰維持と立ち上がりサポートが可能か検証実験を行った.
実験の様子を\figref{equipment}に示す.0から200 kPaの圧縮空気を
印加して変化の様子
\fig{concept.eps}{width=0.9\hsize}{Concept of a wearable chair}
\begin{figure}[tbp]
    \begin{minipage}[b]{0.45\linewidth}
      \centering
      \includegraphics[keepaspectratio, scale=0.17]{/home/osada/template_robomech2024_tex/fig/soutinamae.eps}
      \subcaption{Prototype of a small wearable chair}
      \label{fig:equipment_a}
    \end{minipage}
    \begin{minipage}[b]{0.45\linewidth}
      \centering
      \includegraphics[keepaspectratio, scale=0.25]
      {/home/osada/template_robomech2024_tex/fig/souticoncept.eps}
      \subcaption{Articulated structure of the seat}
      \label{fig:equipment_b}
    \end{minipage}
    \caption{Experimental equipment}
    \label{fig:equipment}
  \end{figure}

  \begin{figure}[tbp]
    \begin{minipage}[b]{0.45\linewidth}
      \centering
      \includegraphics[keepaspectratio, scale=0.35]
      {/home/osada/template_robomech2024_tex/fig/dausamae.eps}
      \subcaption{Non-applied}
    \end{minipage}
    \begin{minipage}[b]{0.45\linewidth}
      \centering
      \includegraphics[keepaspectratio, scale=0.35]
      {/home/osada/template_robomech2024_tex/fig/dausaato.eps}
      \subcaption{At 300 kPa applied}
    \end{minipage}
    \caption{Experiment}
    \label{fig:experiment}
  \end{figure}
\noindent
を調べた.
\figref{equipment}の(a)は非印加時の状態であり,\figref{equipment}の(b)は,
300 kPa印加している状態である.実験結果から300 kPa印加することで
DARwin-OPを持ち上げることができた.
また,その状態でコンプレッサを止め電磁弁を閉じ,エネルギー供給を止めても
その状態を維持することも確認できた.
% \section{空気圧回生機構}
\subsection{空気圧回生機構の検討}
通常の使用した圧縮空気を排気する場合は大気圧解放されている.
しかし,圧縮空気を大気圧解放することは,空気圧シリンダに用いた
圧縮空気のエネルギーをすべて捨てることを意味する.バッテリーのみで
長時間動作を実現するためには,余分な消費電力を減らす,すなわち
コンプレッサの動作時間を短くすることが求められる.
そこであらかじめ大気圧以上に加圧された空気をコンプレッサに
吸気することでコンプレッサの動作時間を短くする.
この空気圧回生を用いた空気回路図を\figref{kaiseizukakinaosi1.eps}にを示す.
通常は排気弁によって大気解放されるアクチュエータ内の圧縮空気の一部を回生弁から
回生用サブタンクに送り込み,一時的に圧縮空気を貯蔵する.
回生用サブタンクに貯蔵された圧縮空気は,コンプレッサが動作する際には
吸気として供給され,圧縮空気の回生動作を実現する.\figref{kaironaosi.eps}
に使用するコンプレッサとタンクを示す.圧縮空気を生成するのに必要な
コンプレッサは,ウェアラブルで移動可能な仕様にするためには小型で,
軽量であることが求められる.コンプレッサは,SQUSE社 MP-2-2-Cで
重さが0.18 kgであり,軽量で,最大圧縮圧は,400 kPaである.
使用するタンクは一般的な炭酸飲料用のペットボトルで容量が500 ml
で耐圧が1.9~2.3 MPaである\cite{petto}.
\subsection{コンプレッサの吸気切り替えに関する検証}
コンプレッサで空気を圧縮する際に,大気圧以上に加圧された回生タンクから
優先的に吸気されることが望ましい.
しかし,大気圧と常に接続されていると回生タンクから吸気されないタイミングが生じる可能性がある.
そこで,圧力の違う2種類のタンクを用意し,そのタンクから吸気をを行い,圧力変化を確認することで,
吸気の切り替えが必要であるかを確認した.\figref{tankidoukaikai.eps}に実験結果を示す.150 kPaと200 kPaの2種類の加圧タンクを用意し,
大気圧から180 kPaまで大気圧から吸気を行い圧縮空気を生成し,その後,二つの加圧タンクから吸気を行い,圧縮空気を生成している.
この実験結果より,コンプレッサの吸気は,最初は圧力が高いほう加圧タンクからのみであったが,同じ圧力になった瞬間に両方の加圧タンクからの吸気に切り替わっていることが確認できた.
また,圧力が高いほうから吸気されるため,コンプレッサの吸気の切り替えは必要ないことが確認できた.
\fig{kaiseizukakinaosi1.eps}{width=1\hsize}
{The pneumatic circuit with the regenerative air
pressure system  and air flows of each states}
\fig{kaironaosi.eps}{width=0.7\hsize}{Circuit created}
% \fig{tankidoukaikai.eps}{width=1\hsize}
{Compressor intake air switching experiment}
\fig{tankmoderu.eps}{width=1\hsize}
{Regenerative mchanism model}
\subsection{シリンダから回生タンクに回収モデル}
\figref{tankmoderu.eps}にシリンダから回生タンクに圧力を回収するモデルを示す.
これは,空気回生機構の一部を簡略化したものであり,シリンダと回生タンクの間は,バルブでとじ
られている.ここでは,シリンダの圧力を$P_1$
,シリンダの体積を$V_1$,回生タンクの圧力を$P_2$,回生タンクの体積を$V_2$とする.
バルブを開けた時の圧力を$P$は,
\begin{eqnarray}
    P = \frac{1}{V_{1} + V_{2}}(P_{1}V_{1} + P_{2}V_{2})
\end{eqnarray}
で表される.
\fig{tankidoukai.eps}{width=1\hsize}
{Experiment of solenoid valve switching conditions by pressure}
\fig{kaiseikai.eps}{width=1\hsize}
{Compression speed comparison experiment with regeneration}
\subsection{シリンダから回生タンクへの圧力回収実験}
\figref{kaiseizukakinaosi1.eps}の横線で示すシリンダから回生タンクへ排気したときに回収できる圧力の計測実験を行った.
実験の結果を\figref{tankidoukai.eps}に示す.
シリンダの圧力が200 kPaの時に,回生タンクが109 kPaであり,5 kPa回収,
シリンダ圧力が,300 kPaの時に,回生タンク圧力が,114 kPa,で9 kPa回収することが確認できた.
(1)式を用いて計算すると,200 kPaの時に,106 kPa,300 kPaの時に,110 kPa,となり,近い値
であることが確認することができた.
実験の値と計算式の値の誤差として考えられるのは,ポリウレタンチューブの圧力が含まれていないことや,
圧力センサの性能による影響があると考えられる.
また,シリンダの圧力が高いほうが,シリンダと回生タンクが同じ圧力
になるまで時間を要することが分かる.
\subsection{コンプレッサ圧縮速度の短縮実験}
メインタンクの目標圧力を300 kPaとし,大気圧から
圧縮空気を生成する場合とサブタンクに200 kPaが
印加された状態との比較実験を行った.
実験の結果を\figref{kaiseikai.eps}に示す.
この結果より,空気回生システムを用いた場合と用いなかった
場合とを比べると,回生システムを用いた方がタンク内圧の回復が
早いことが分かる.この条件では,目標圧力300 kPaを生成する
時間を約10 sほど短縮できていることが確認することができた.
% \section{結言および今後の展望}
小型のウェアラブル椅子を製作し,
立ち上がり時支援とエネルギーを消費なしの
中腰支援を行うことができた.
また,空気圧回生機構により圧縮空気を生成する時間を
短縮できることでコンプレッサの使用時間を減らせ,
長時間動作に貢献することができることを確認した.
今後の展望としては,ウェアラブル椅子を実際の人間の大きさで製作する.
その後,中腰支援と立ち上がりサポートが行うことができるかを現場で検証する.
% \section{結論}
本論文では,伸縮するブラキエーションロボットを提案し,実機を製作した.
また,伸縮による重心移動のメカニズムを解明し,それを基に実機でセンサフィードバックによる振動拡大を実現した.
更に,伸縮を利用したブラキエーション動作を実機で実現し,空中相を含まずに遠くのバーへ到達できることを示した.
今後の展望として,伸縮するブラキエーションロボットにモデル予測制御を導入し,次のバーへ自律的に移動することや,空中相を含むブラキエーションを行うことなどがある.
\section{緒言}
農場や工場などでの作業では,収穫や製造の各工程において長時間にわたる中腰姿勢を維持したり,
立ち上がり動作が求められることが多く身体的負担が大きい.そのため,身体的負担を軽減するために様々
なアシスト装置の研究が行われている
\cite{power2}\cite{every}\cite{HAL}\cite{BLEEX}.
アシスト装置は,動作の補助・増強をすることで身体的負担を軽減する.
アシスト装置の中で特に中腰姿勢を維持する装置としてウェアラブルチェアがある.
Yao TuらのE-legは,電磁スイッチをロックすることで,
0.4 mから0.8 mの間で中腰姿勢維持が可能である\cite{yao}.Noonee社のChairless ChairやArchelis,永島らの装置は,
固定高さでの中腰姿勢をサポートを行うことが可能である\cite{noonee}\cite{osada1}\cite{kinki}.
これらの装置が適用される,農場や工場での作業は野外や移動範囲が広い場合が多く,
電源を得ることが困難であるため,長時間動作が求められる.しかし,これらは,
バッテリー駆動であるため動作時間が制限されることや,固定高さでの作業にしか対応できず,
立ち上がり動作を支援することができないなどの問題がある.一方,Magdumらの研究では,
ウェアラブルチェアの高さ調整に空気圧シリンダを利用することで,中腰支援と立ち上がりが
可能である.空気圧シリンダに圧縮空気を印加し,弁を閉じることでエネルギー消費なしに
中腰姿勢を維持することが可能である\cite{nita}.
しかし,この空気圧システムでは,使用した圧縮空気を大気圧解放するためエネルギーが
無駄になっている.このエネルギーの無駄を削減するための方法として,
一度使用した圧縮空気を再利用する空気圧回生機構が存在
する\cite{kuma}\cite{sasaki}.\\
 本研究では,空気を動力源とすることで立ち上がり時支援とエネルギー消費なしの
中腰支援の両方が可能なコンプレッサ搭載型のウェアラブル椅子を製作し,身体的負担を軽減する.
さらに空気圧回生機構を組み込むことにより使用した圧縮空気を排気する装置と比較して長時間動作
を実現する.
\section{ウェアラブル椅子のコンセプトと空気回生機構}
% \subsection{はじめに}
% 本章ではウェアラブル椅子のコンセプトと空気回生機構の有効性について述べる.
\begin{figure}[h]%
        \begin{center}%
         \pastefig{\FIGDIR/2.1.eps}{width=0.8\hsize}%
         \vspacebeforefig
         \caption{Concept of wearable chair}%
         \figlabel{2.1.eps}%
         \vspace{-5mm}
         \vspaceafterfig
        \end{center}%
       \end{figure}%
\begin{figure}[h]%
        \begin{center}%
         \pastefig{\FIGDIR/kaisei(zu2).eps}{width=1\hsize}%
         \vspacebeforefig
         \caption{The pneumatic circuit with the regenerative air
         pressure system  and air flows of each states}%
         \figlabel{kaisei(zu2).eps}%
         \vspace{-8mm}
         \vspaceafterfig
        \end{center}%
\end{figure}%
%     \fig{kaisei(zu).eps}{width=0.9\hsize}
%     {The pneumatic circuit with the regenerative air
%     pressure system  and air flows of each states}
\subsection{ウェアラブル椅子のコンセプト}
\figref{2.1.eps}は,本研究で目的とする動作を示す.ウェアラブル椅子は,
人間の臀部付近に空気圧シリンダが設置された構造をしている.立ち上がり時支援と
中腰支援の2つの動作を行うことができる.
\figref{2.1.eps}aと\figref{2.1.eps}bは中
腰支援の様子である.
シリンダが一番縮んだの状態から空気を印加し,弁を閉じることで\figref{2.1.eps}bで示す
任意の位置での中腰支援が可能である.また.空気を動力源としているため
空気を印加し弁を閉じればエネルギー消費無しに中腰支援を行うことが可能である.
\figref{2.1.eps}cは立ち上がり時支援の様子である.空気を印加し,シリンダを伸ばすことで,
立ち上がり時支援を行うことができる.\figref{2.1.eps}dは移動中の様子である.
使用者が移動する間,シリンダと座面とのなす角を変えることで移動中に
農作物などにシリンダが接触を防ぐことができる.
\subsection{空気回生機構の有効性}
通常,使用した圧縮空気を排気する場合は大気へと解放されている.
しかし,圧縮空気を大気へと解放することは,空気圧シリンダに用いた
圧縮空気のエネルギーをすべて捨てることを意味する.バッテリーのみで
長時間動作を実現するためには,余分な消費電力を減らす,すなわちコンプレッサ
の動作時間を短くすることが求められる.そこであらかじめ大気圧以上に加圧された空気
をコンプレッサに吸気することでコンプレッサの動作時間を短くする.
この空気圧回生を用いた空気回路図を\figref{kaisei(zu2).eps}にを示す.通常は排気弁によって
大気解放されるアクチュエータ内の圧縮空気の一部を回生弁から回生用回生タンクに送り込み,
一時的に圧縮空気を貯蔵する.回生用回生タンクに貯蔵された圧縮空気は,
コンプレッサが動作する際には吸気として供給され,圧縮空気の回生動作を実現する.
\begin{figure}[t]%
        \begin{center}%
         \pastefig{\FIGDIR/wearablechair.eps}{width=0.3\hsize}%
         \vspacebeforefig
         \caption{Wearable chair}%
         \figlabel{wearablechair.eps}%
         \vspace{-6mm}
         \vspaceafterfig
 \end{center}%
\end{figure}% 
% \begin{figure}[t]%
%         \begin{center}%
%          \pastefig{\FIGDIR/4.7.eps}{width=0.2\hsize}%
%          \vspacebeforefig
%          \caption{Compressor(SQUSE,MP-2-2-C)}%
%          \figlabel{4.7.eps}%
%          \vspaceafterfig
% \end{center}%
% \end{figure}% 
%    \fig{wearablechair.eps}{width=0.27\hsize}
%    {Wearable chair}
%    \fig{4.7.eps}{width=0.2\hsize}{Compressor(SQUSE,MP-2-2-C)}
\begin{figure}[tbp]
    \begin{minipage}[b]{0.45\linewidth}
      \centering
      \includegraphics[keepaspectratio, scale=0.2]
      {/home/osada/shuuron/1_29/shuuron/fig/eqipment(stand).eps}
      \subcaption{Standing}
    \end{minipage}
    \begin{minipage}[b]{0.45\linewidth}
            \centering
            \includegraphics[keepaspectratio, scale=0.2]
            {/home/osada/shuuron/1_29/shuuron/fig/eqipment(sit).eps}
            \subcaption{Seating}
    \end{minipage}
    \caption{Two states of the wearable chair}
    \vspace{-4mm}
    \label{fig:3.7}
\end{figure}
\section{長時間動作が可能なウェアラブル椅子の開発}
\subsection{ウェアラブル椅子の開発}
コンセプトを基に開発したウェアラブル椅子を\figref{wearablechair.eps}に示す.
ウェアラブル椅子の重量は,5.7 kgであり,シリンダ非伸長時の幅・奥行・高さは,
425 × 360 × 420 mm,シリンダ伸長時の幅・奥行・高さは,425 × 360 × 620 mm である.
シリンダはコガネイ社製のCCDA50X200を使用している.
炭酸飲料用の1 Lのペットボトルを圧力を供給するメインタンクに3本,空気回生機構に
使用する回生タンクに3本使用した.
圧縮空気を生成するのに必要なコンプレッサは,ウェアラブル
で移動可能な仕様にするために小型かつ軽量であることが求められる.コンプレッサは
SQUSE社 MP-2-2-Cを用いている.重さは0.194 kgであり,最大圧縮圧は,400 kPaである.
ウェアラブル椅子の構造はアルミフレームで構成されており,関節構造を有している.
関節構造を設けることにより,移動の時は,エアシリンダと座面の角度を変更することが可能となり,
使用者がウェアラブル椅子を着用したまま移動の妨げにならないような構造となっている.
また,ロックピンを差し込むことでにシリンダと座面とのなす角を90度に固定することが可能である.
\subsection{ウェアラブル椅子の装着}
\figref{3.7}(a)にウェアラブル椅子を装着した状態で立っている様子を示す.
ウェアラブル椅子と使用者は,Techouter社製のハーネスと,東洋物産工業社製の
バックル付きベルトで固定されている.\figref{3.7}(b)にロックピンを差し込み,
シリンダと座面の角度を90度に固定して座った状態を示す.\figref{3.7}(a)(b)に
示すように使用者にウェアラブル椅子を身に装着可能である.
\section{中腰支援と空気回生実験}
\subsection{ウェアラブル椅子の沈み量}\subseclabel{hl}
ウェアラブル椅子で中腰支援を行う場合,立ち上がった状態でシリンダの伸ばす部屋と
シリンダを縮める部屋に空気を印加し.その後人間が座ることで中腰支援を行う.
中腰支援時のシリンダの概略図を\figref{4.3(2).eps}に示す.シリンダを伸ばす部屋
の初期圧力,体積,断面積,長さはそれぞれ
$P_{h\_cu1}$,$V_{h\_cu1}$,$S_{cu}$,$L_{cu1}$,
シリンダを縮める部屋の初期圧力,体積,断面積,長さはそれぞれ
$P_{h\_cb1}$,$V_{h\_cb1}$,$S_{cb}$,$L_{cb1}$とする.
質量$M$の物体を乗せた時,シリンダが$z_{h1}$だけ縮んだとする.
このとき,シリンダを伸ばす部屋の圧力,体積はそれぞれ$P_{h\_cu2}$,$V_{h\_cu3}$,
シリンダを縮める部屋の圧力,体積はそれぞれ$P_{h\_cb2}$,$V_{h\_cb2}$に変化したとする.
等温変化とするとボイルの法則からと縮めた後のピストンの圧力は
\begin{eqnarray}
        P_{h\_cu1}S_{su}L_{cu1} &=& P_{h\_cu2}S_{cu}(L_{cu1} - z_{h1}) \label{6} \\
        P_{h\_cb1}S_{cb}L_{cb1} &=& P_{h\_cb2}S_{cb}(L_{cb1} + z_{h1}) \label{7}
\end{eqnarray}
となる.このとき,シリンダに作用する力は釣り合い
\begin{eqnarray}
       P_{h\_cu2}S_{cu} - P_{h\_cb2}S_{cb}  &=& Mg\label{8}
\end{eqnarray}
となる.したがって,式\eqref{6},式\eqref{7},式\eqref{8}を基に沈み量$z_{h1}$
は求められる.
% \fig{standup pressure.eps}{width=0.8\hsize}{Stand-up assistance experiment (pressure)}
% \clearpage
\fig{halfsit pressure.eps}{width=0.65\hsize}{Half-sitting posture support experiment(pressure)}
\fig{halfsit haiki(2).eps}{width=0.65\hsize}
{Sub-tank recovery model during exhausting air from bottom room}
\subsection{ウェアラブル椅子の中腰支援}
ウェアラブル椅子に人間が乗せた状態で中腰支援が可能か確認を行った.
人が立ち上がった状態で,シリンダが最も短い状態の時に,伸ばす部屋と縮める部屋に
圧縮空気を印加し,弁を閉じた.メインタンクの容量は3 L,400 kPa印加されている.
圧力変化を\figref{halfsit pressure.eps}示す.始めに,人を乗せずにメインタンクから
シリンダに350 kPa印加し,弁を閉じた.この時,シリンダを伸ばす部屋の圧力が350 kPa,
シリンダを縮める部屋の圧力は409 kPa,シリンダは115 mm伸びた状態となる.シリンダを
伸ばす部屋とシリンダを縮める部屋の圧力が違うのは,ロッドで断面積が違うためである.
次に,ウェアラブル椅子に座った.この時,約46.8 kgの荷重がかかり,
シリンダの長さは115 mmから90 mmに変化し,25 mm沈んだ.また,
\figref{halfsit pressure.eps}より,26 s付近で人がウェアラブル椅子に座ると,
シリンダの伸ばす部屋の圧力が高くなり,シリンダの縮める部屋の圧力は低下した.
また,\subsecref{hl}より求められるシリンダの沈み量$z_{h1}$は22 mmであり,
沈み量の実験値と近い値であることが確認できた.加えて,中腰支援中はシリンダに空気を
印加し弁を閉じるため,エネルギー消費なしのサポートができていると考えられる.そのため,
中腰支援は,同じ高さで作業を行う場合,1度のメインタンクの圧縮空気の使用のみでよいため,
長時間動作に貢献することが可能である.
% \subsection{立ち上がり支援後のシリンダから回生タンクの圧力平衡}
%         立ち上がり支援を行った後回生タンクに回収するシリンダと回生タンクの概略図を\figref{4.2.eps}に示す.シリンダで立ち上がり時支援を行った後,シリンダを伸ばす部屋の排気し,回生タンクに圧力を回収する.シリンダを縮める部屋は,弁を閉める.シリンダには質量$M$の物体が乗っていると想定する.シリンダを伸ばす部屋の初期圧力,体積,断面積はそれぞれ,$P_{cu2}$,$V_{cu2}$,$S_{cu}$で,シリンダを縮める部屋の初期圧力,体積,断面積はそれぞれ,$P_{cb2}$,$V_{cb2}$,$S_{cb}$とする.回生タンクの圧力,体積はそれぞれ,$P_{sub1}$,$V_{sub1}$とする.シリンダを伸ばす部屋から回生タンクに空気を排気すると,シリンダの伸ばす部屋と回生タンクの圧力は平衡状態となる.このとき,シリンダが$z_{2}$だけ伸びたとすると,ピストンは,シリンダを伸ばす部屋の圧力,体積はそれぞれ,$P_{eq2}$,$V_{cu3}$,シリンダの縮める部屋の圧力,体積は,$P_{cb3}$,$V_{cb3}$,回生タンクの圧力がメインタンクの圧力は,$P_{eq2}$に変化したとする.ここで平衡状態の圧力$P_{eq2}$は
%         \begin{eqnarray}
%                 P_{cu2}V_{cu2} + P_{s1}V_{s} &=& P_{eq2}V_{cu3} + P_{eq2}V_{s}\label{9}\\
%                 V_{cu3} &=& V_{cu2} - S_{cu}z_{2}
%         \end{eqnarray}   
%         また,平衡状態でのシリンダに作用する力は釣り合い,
%         \begin{eqnarray}
%                 P_{eq2}S_{cu} &=& P_{cb3}S_{cb} + Mg\label{11}\\
%                 P_{cb2}V_{cb2} &=& P_{cb3}V_{cb3} \label{12}\\
%                 V_{cb3} = V_{cb2} + (S_{cb}z_{2})
%         \end{eqnarray}
%         ここで,式\eqref{12}よりシリンダを縮める部屋の空気は,ボイルの法則を満たす.式\eqref{9}と式\eqref{11}を基にシリンダの伸び$z_{2}$と$P_{eq2}$を求めることができる.
 % \fig{halfsitstand.eps}{width=1\hsize}
 % {Sub-tank recovery model (bottom room exhaust)}
 \subsection{中腰支援後のシリンダから回生タンクの圧力平衡}\subseclabel{h2}
 中腰支援を行った後,シリンダを縮める部屋の圧力を回生タンクに排気することで,
 シリンダの縮めるの力が弱まり,立ち上がり支援することが可能である.
 シリンダの縮める部屋の圧力を排気する際のシリンダとメインタンクの概略図を
 \figref{halfsit haiki(2).eps}に示す.
 シリンダには,質量$M$の物体が乗っていると想定する.
 シリンダを伸ばす部屋の初期圧力,体積,断面積はそれぞれ,
 $P_{h\_cu2}$,$V_{h\_cu2}$,$S_{cu}$で,シリンダの縮める部屋の初期圧力,
 体積,断面積は,$P_{h\_cb2}$,$V_{h\_cb2}$,$S_{cb}$とする.回生タンクの圧力,
 体積はそれぞれ,$P_{h\_s1}$,$V_{s}$とする.シリンダを縮める部屋から回生タンクに
 空気を排気し,シリンダの縮める部屋と回生タンクの圧力は平衡状態となる.
 シリンダが$z_{h2}$だけ伸びたとすると,ピストンは,シリンダを伸ばす部屋の圧力,
 体積はそれぞれ,$P_{h\_cu3}$,$V_{h\_cu3}$,シリンダの縮める部屋の圧力,
 体積はそれぞれ,$P_{h\_eq1}$,$V_{h\_cb3}$,回生タンクの圧力がメインタンクの圧力
 は,$P_{h\_eq1}$に変化したとする.ここで,平衡状態の圧力$P_{h\_eq1}$は
\begin{eqnarray}
         P_{h\_cb2}V_{h\_cb2} + P_{h\_s1}V_{s} &=& P_{h\_eq1}V_{h\_cb3} + P_{h\_eq1}V_{s}\label{14}\\
         V_{h\_cb3} &=& V_{h\_cb2} - S_{cb}z_{h2}
\end{eqnarray}
また,平衡状態でのシリンダに作用する力は釣り合い,
\begin{eqnarray}
         P_{h\_eq1}S_{cb} &=& P_{h\_cu3}S_{cu} - Mg\label{16}\\
         P_{h\_cu2}V_{h\_cu2} &=& P_{h\_cu3}V_{h\_cu3}\label{17}\\
         V_{h\_cu3} &=& V_{h\_cu2} + (S_{cu}z_{h2})
\end{eqnarray}
ここで,式\eqref{17}よりシリンダを縮める部屋の空気は,
ボイルの法則を満たす.式\eqref{14}と式\eqref{16}を基にシリンダの
伸び$z_{h2}$と$P_{h\_eq1}$を求めることができる.
        % \figref{standup kaisei(noregeneration).eps}よりメインタンクの圧力を回復させるのに約8.0 min要した.次に一度立ち上がり支援を行った後, 3 Lの回生タンクに圧力を回収し,回生タンクの空気を使用して空気を圧縮する実験を行った.実験結果を \figref{standup kaisei(regeneration).eps}に示す.実験の結果から約4.9 minかかった.これらの実験結果から回生タンクを用いるとコンプレッサの動作時間を約3.1 min削減することが確認できた. 
        % \fig{halfsit-stand.eps}{width=1\hsize}
        % {Assistance in standing up after half-sit posture support}
        % \fig{halfsit length.eps}{width=0.7\hsize}
        % {Stand-up assistance experiment after half-sitting posture (cylinder length and weight)}
        % \fig{halfsit haiki.eps}{width=0.8\hsize}
        % {Transition in recovery pressure after half sitting posture support}    
        \begin{figure}[t]%
                \begin{center}%
                 \pastefig{\FIGDIR/halfsit haiki.eps}{width=0.65\hsize}%
                 \vspacebeforefig
                 \caption{Transition in recovery pressure after half sitting posture support}%
                 \figlabel{halfsit haiki.eps}%
                 \vspaceafterfig
                \end{center}%
        \end{figure}%  
        \begin{figure}[t]%
                \begin{center}%
                 \pastefig{\FIGDIR/halfsit-stand.eps}{width=0.65\hsize}%
                 \vspacebeforefig
                 \caption{Assistance in standing up after half-sit posture support}%
                 \figlabel{halfsit-stand.eps}%
                 \vspaceafterfig
                \end{center}%
        \end{figure}%  
        \subsection{中腰支援後の回生タンクへの回収実験}
         中腰支援を行った後,シリンダの縮める部屋から回生タンクへ一度排気したとき
         に回収できる圧力を確認した.実験の結果を\figref{halfsit haiki.eps}に示す.
         \figref{halfsit haiki.eps}より,シリンダを縮める部屋の圧力が296 kPaの時に
         回生タンクへ排気すると,回生タンクに圧力を21 kPa回収できた.\subsecref{h2}の式
         を用いて計算すると,シリンダを縮める部屋の圧力が296 kPaの時に回生タンクへ回収
         できる圧力$P_{h\_eq1}$は21 kPaであり,実験値と近い値であることが確認できた.
         また,立ち上がりの様子を\figref{halfsit-stand.eps}に,立ち上がり時のシリンダ
         の長さと重量の変化を\figref{halfsit length.eps}にを示す.
         シリンダには平均45.7 kgの荷重がかかっている.シリンダを縮める部屋の
         圧力が低下することにより,立ち上がり支援を行うことができ,90 mmから147 mmまで,
         57mmの立ち上がり支援を行うことが確認できた.
         \subsecref{h2}をの式用いてシリンダの伸び$z_{h2}$を計算すると,
         シリンダの長さは65 mm であり,少し実験値と差があることが確認された.
         理論値より,実験値がほうが小さいと考えられるのは,粘性の影響が考えられる.
         \begin{figure}[t]%
                \begin{center}%
                 \pastefig{\FIGDIR/halfsit length.eps}{width=0.7\hsize}%
                 \vspacebeforefig
                 \caption{Stand-up assistance experiment after half-sitting posture (cylinder length and weight)}%
                 \figlabel{halfsit length.eps}%
                 \vspaceafterfig
                \end{center}%
        \end{figure}% 
        \begin{figure}[t]%
                \begin{center}%
                 \pastefig{\FIGDIR/halfsit kaisei0.eps}{width=0.65\hsize}%
                 \vspacebeforefig
                 \caption{Transition in pressure for compression speed comparison (without regeneration, after
                 half-sitting posture support)}%
                 \figlabel{halfsit kaisei0.eps}%
                 \vspaceafterfig
                \end{center}%
        \end{figure}% 
        \begin{figure}[t]%
                \begin{center}%
                 \pastefig{\FIGDIR/halfsit kaisei1.eps}{width=0.65\hsize}%
                 \vspacebeforefig
                 \caption{Transition in pressure for compression speed comparison (with regeneration, after half-
                 sitting posture support)}%
                 \figlabel{halfsit kaisei1.eps}%
                 \vspaceafterfig
                \end{center}%
        \end{figure}% 
 \subsection{中腰支援後のコンプレッサ圧縮速度短縮実験}
中腰支援を一回行った後,空気回生の有無により,メインタンクの圧力
を回復させる時間を短縮できるか確認を行った.空気回生をせずに大気から
空気を取り込んでメインタンクの圧力を400kPaまで回復させた結果
を\figref{halfsit kaisei0.eps}に,回生タンクから空気を取り込んで
メインタンクの圧力を回復させた結果を\figref{halfsit kaisei1.eps}に示す.
\figref{halfsit kaisei0.eps}より回生なしの場合はメインタンクの圧力を回復
させるのに約9.5 min要した.一方,\figref{halfsit kaisei1.eps}より回生ありの
場合は約8.2 min要した.これにより,回生タンクを用いるとコンプレッサの動作時間
を約1.3 min削減することが確認できた.
\begin{figure}[t]%
        \begin{center}%
         \pastefig{\FIGDIR/halfsit kaisei0(1).eps}{width=0.8\hsize}%
         \vspacebeforefig
         \caption{Transition in pressure for compression speed comparison (without regeneration, after
         half-sitting posture support)}%
         \figlabel{halfsit kaisei0(1).eps}%
         \vspace{-7mm}
         \vspaceafterfig
        \end{center}%
\end{figure}% 
\begin{figure}[t]%
        \begin{center}%
         \pastefig{\FIGDIR/halfsit kaisei1(1).eps}{width=0.8\hsize}%
         \vspacebeforefig
         \caption{Transition in pressure for compression speed comparison (with regeneration, after half-
         sitting posture support)}%
         \figlabel{halfsit kaisei1(1).eps}%
         \vspace{-9mm}
         \vspaceafterfig
        \end{center}%
\end{figure}% 
\begin{figure}[t]%
        \begin{center}%
         \pastefig{\FIGDIR/support standup(1).eps}{width=0.8\hsize}%
         \vspacebeforefig
         \caption{Stand-up assistance experiment flow}%
         \figlabel{support standup(1).eps}%
         \vspace{-8mm}
         \vspaceafterfig
        \end{center}%
\end{figure}% 
\begin{figure}[t]%
        \begin{center}%
         \pastefig{\FIGDIR/standup pressure(1).eps}{width=0.8\hsize}%
         \vspacebeforefig
         \caption{Stand-up assistance experiment (pressure)}%
         \figlabel{standup pressure(1).eps}%
         \vspace{-8mm}
         \vspaceafterfig
        \end{center}%
\end{figure}% 
\begin{figure}[t]%
        \begin{center}%
         \pastefig{\FIGDIR/standup support length(1).eps}{width=0.8\hsize}%
         \vspacebeforefig
         \caption{Stand-up assistance experiment (cylinder length and weight)}%
         \figlabel{standup support length(1).eps}%
         \vspace{-8mm}
         \vspaceafterfig
        \end{center}%
\end{figure}%
\subsection{立ち上がり支援実験とシリンダの伸縮(伸ばす部屋印加)}
        開発したウェアラブル椅子に人を乗せた状態で立ち上がり支援可能か確認を行った.
        シリンダが一番短い状態で,シリンダの伸ばす部屋に空気を印加した.
        シリンダの縮める部屋は弁を閉じている.荷重は,体重計にウェアラブル椅子を置き,
        計測を行っている.伸びた長さはカメラ画像から求めた.メインタンクの容量は3 Lであり,
        400 kPa印加されている.実験の様子を\figref{support standup(1).eps}に示す.
        メインタンクから圧力を供給したときの圧力変化を\figref{standup pressure(1).eps}に,
        \figref{standup support length(1).eps}にシリンダ長さと荷重の変化を示す.
        メインタンクからシリンダに空気を印加したところ,
        シリンダを伸ばす部屋の圧力は362 kPa,平均43.0 kgの荷重がかかっている状態
        で129 mm持ち上げることを確認した.また,\subsecref{h2}より
        求められるメインタンクが400 kPaから平衡状態へ移行した時のシリンダを伸ばす
        部屋の圧力$P_{eq}$は360 kPaであり,測定値と近い値を取ることが確認できた.
        加えて,\subsecref{h2}より求められるシリンダの伸び$z$は131 mmであり,
        測定値と近い値を取ることを確認できた. 
\section{結  言}
本研究では,中腰支援と立ち上がり支援を行うことができるとともに空気回生機構による
長時間動作を行うことができるウェアラブル椅子の開発を目標とした.
まず,ウェアラブル椅子のコンセプトを提案した.提案したコンセプトは,
人間の臀部にシリンダがついており,空気圧を動力源として中腰支援と立ち上がり支援を行う.
また,空気回生機構の有効性について述べた.ウェアラブル椅子を製作した.
製作したウェアラブル椅子をハーネスとバックル付きベルトで使用者に装着することで
移動することができる.製作したウェアラブル椅子を用いて,立ち上がり支援と中腰支援を
行うことができることを確認した.また,空気回生機構を用いることで,
コンプレッサの動作時間が短縮され,エネルギー消費量が少なくなることを確認した.\\
 今後の展望として,ウェアラブル椅子を実際の農業現場で使用することで,
ウェアラブル椅子の有効性を確認する.


\nocite{*}

\bibliography{reference}
\bibliographystyle{junsrt}

\normalsize
\end{document}
