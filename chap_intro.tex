\section{緒言}
作業における長時間の中腰姿勢作業や立ち上がり動作は身体的負担が
大きい.身体的負担を軽減するためにアシスト装置の研究が行われている.
これらの作業は野外での場合が多く,電源を得ることが困難であるため,
バッテリーでの長時間動作が求められる.
中腰姿勢作業の負担軽減を行うために様々なウェアラブル椅子が研究されている\cite{noonee}\cite{yao}\cite{osada1}\cite{isu}.
これらは,中腰支援維持をすることが
可能であるが,立ち上がり支援を行うことができない.Magdumらの研究では,
ウェアラブルな椅子の高さ調整に空気圧シリンダを利用することで,中腰支援と
立ち上がりが可能である.空気圧シリンダに圧縮空気を印加し,弁を閉じることで
中腰姿勢維持をエネルギー消費なしに中腰姿勢を維持することが可能である\cite{nita}.
しかし,使用した圧縮空気を大気圧解放するためエネルギーが無駄になっている.
このエネルギーの無駄を省くため,一度使用した圧縮空気を再利用する空気圧
回生機構が存在する\cite{kumakura}\cite{kuma}\cite{sasaki}.\\
 本研究では,立ち上がり時支援と中腰支援が可能な空気を動力源
とするウェアラブル椅子を製作し,身体的負担を軽減するとともに,空気圧回生機構を
組み込むことにより長時間動作を実現する.
% noonee社のChairless Chair 2.0\cite{noonee}や
% アルケリスは\cite{osada1},ウェアラブルな椅子であり,