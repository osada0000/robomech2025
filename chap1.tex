\section{緒言}
農場や工場などでの作業では,収穫や製造の各工程において長時間にわたる中腰姿勢を維持したり,
立ち上がり動作が求められることが多く身体的負担が大きい.そのため,身体的負担を軽減するために様々
なアシスト装置の研究が行われている
\cite{power2}\cite{every}\cite{HAL}\cite{BLEEX}.
アシスト装置は,動作の補助・増強をすることで身体的負担を軽減する.
アシスト装置の中で特に中腰姿勢を維持する装置としてウェアラブルチェアがある.
Yao TuらのE-legは,電磁スイッチをロックすることで,
0.4 mから0.8 mの間で中腰姿勢維持が可能である\cite{yao}.Noonee社のChairless ChairやArchelis,永島らの装置は,
固定高さでの中腰姿勢をサポートを行うことが可能である\cite{noonee}\cite{osada1}\cite{kinki}.
これらの装置が適用される,農場や工場での作業は野外や移動範囲が広い場合が多く,
電源を得ることが困難であるため,長時間動作が求められる.しかし,これらは,
バッテリー駆動であるため動作時間が制限されることや,固定高さでの作業にしか対応できず,
立ち上がり動作を支援することができないなどの問題がある.一方,Magdumらの研究では,
ウェアラブルチェアの高さ調整に空気圧シリンダを利用することで,中腰支援と立ち上がりが
可能である.空気圧シリンダに圧縮空気を印加し,弁を閉じることでエネルギー消費なしに
中腰姿勢を維持することが可能である\cite{nita}.
しかし,この空気圧システムでは,使用した圧縮空気を大気圧解放するためエネルギーが
無駄になっている.このエネルギーの無駄を削減するための方法として,
一度使用した圧縮空気を再利用する空気圧回生機構が存在
する\cite{kuma}\cite{sasaki}.\\
 本研究では,空気を動力源とすることで立ち上がり時支援とエネルギー消費なしの
中腰支援の両方が可能なコンプレッサ搭載型のウェアラブル椅子を製作し,身体的負担を軽減する.
さらに空気圧回生機構を組み込むことにより使用した圧縮空気を排気する装置と比較して長時間動作
を実現する.