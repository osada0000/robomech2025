\begin{figure}[tbp]
    \begin{minipage}[b]{0.45\linewidth}
      \centering
      \includegraphics[keepaspectratio, scale=0.2]
      {/home/osada/shuuron/1_29/shuuron/fig/eqipment(stand).eps}
      \subcaption{Standing}
    \end{minipage}
    \begin{minipage}[b]{0.45\linewidth}
            \centering
            \includegraphics[keepaspectratio, scale=0.2]
            {/home/osada/shuuron/1_29/shuuron/fig/eqipment(sit).eps}
            \subcaption{Seating}
    \end{minipage}
    \caption{Two states of the wearable chair}
    \vspace{-4mm}
    \label{fig:3.7}
\end{figure}
\section{長時間動作が可能なウェアラブル椅子の開発}
\subsection{ウェアラブル椅子の開発}
コンセプトを基に開発したウェアラブル椅子を\figref{wearablechair.eps}に示す.
ウェアラブル椅子の重量は,5.7 kgであり,シリンダ非伸長時の幅・奥行・高さは,
425 × 360 × 420 mm,シリンダ伸長時の幅・奥行・高さは,425 × 360 × 620 mm である.
シリンダはコガネイ社製のCCDA50X200を使用している.
炭酸飲料用の1 Lのペットボトルを圧力を供給するメインタンクに3本,空気回生機構に
使用する回生タンクに3本使用した.
圧縮空気を生成するのに必要なコンプレッサは,ウェアラブル
で移動可能な仕様にするために小型かつ軽量であることが求められる.コンプレッサは
SQUSE社 MP-2-2-Cを用いている.重さは0.194 kgであり,最大圧縮圧は,400 kPaである.
ウェアラブル椅子の構造はアルミフレームで構成されており,関節構造を有している.
関節構造を設けることにより,移動の時は,エアシリンダと座面の角度を変更することが可能となり,
使用者がウェアラブル椅子を着用したまま移動の妨げにならないような構造となっている.
また,ロックピンを差し込むことでにシリンダと座面とのなす角を90度に固定することが可能である.
\subsection{ウェアラブル椅子の装着}
\figref{3.7}(a)にウェアラブル椅子を装着した状態で立っている様子を示す.
ウェアラブル椅子と使用者は,Techouter社製のハーネスと,東洋物産工業社製の
バックル付きベルトで固定されている.\figref{3.7}(b)にロックピンを差し込み,
シリンダと座面の角度を90度に固定して座った状態を示す.\figref{3.7}(a)(b)に
示すように使用者にウェアラブル椅子を身に装着可能である.