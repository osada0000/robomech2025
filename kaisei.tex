\section{空気圧回生機構}
\subsection{空気圧回生機構の検討}
通常の使用した圧縮空気を排気する場合は大気圧解放されている.
しかし,圧縮空気を大気圧解放することは,空気圧シリンダに用いた
圧縮空気のエネルギーをすべて捨てることを意味する.バッテリーのみで
長時間動作を実現するためには,余分な消費電力を減らす,すなわち
コンプレッサの動作時間を短くすることが求められる.
そこであらかじめ大気圧以上に加圧された空気をコンプレッサに
吸気することでコンプレッサの動作時間を短くする.
この空気圧回生を用いた空気回路図を\figref{kaiseizukakinaosi1.eps}にを示す.
通常は排気弁によって大気解放されるアクチュエータ内の圧縮空気の一部を回生弁から
回生用サブタンクに送り込み,一時的に圧縮空気を貯蔵する.
回生用サブタンクに貯蔵された圧縮空気は,コンプレッサが動作する際には
吸気として供給され,圧縮空気の回生動作を実現する.\figref{kaironaosi.eps}
に使用するコンプレッサとタンクを示す.圧縮空気を生成するのに必要な
コンプレッサは,ウェアラブルで移動可能な仕様にするためには小型で,
軽量であることが求められる.コンプレッサは,SQUSE社 MP-2-2-Cで
重さが0.18 kgであり,軽量で,最大圧縮圧は,400 kPaである.
使用するタンクは一般的な炭酸飲料用のペットボトルで容量が500 ml
で耐圧が1.9~2.3 MPaである\cite{petto}.
\subsection{コンプレッサの吸気切り替えに関する検証}
コンプレッサで空気を圧縮する際に,大気圧以上に加圧された回生タンクから
優先的に吸気されることが望ましい.
しかし,大気圧と常に接続されていると回生タンクから吸気されないタイミングが生じる可能性がある.
そこで,圧力の違う2種類のタンクを用意し,そのタンクから吸気をを行い,圧力変化を確認することで,
吸気の切り替えが必要であるかを確認した.\figref{tankidoukaikai.eps}に実験結果を示す.150 kPaと200 kPaの2種類の加圧タンクを用意し,
大気圧から180 kPaまで大気圧から吸気を行い圧縮空気を生成し,その後,二つの加圧タンクから吸気を行い,圧縮空気を生成している.
この実験結果より,コンプレッサの吸気は,最初は圧力が高いほう加圧タンクからのみであったが,同じ圧力になった瞬間に両方の加圧タンクからの吸気に切り替わっていることが確認できた.
また,圧力が高いほうから吸気されるため,コンプレッサの吸気の切り替えは必要ないことが確認できた.
\fig{kaiseizukakinaosi1.eps}{width=1\hsize}
{The pneumatic circuit with the regenerative air
pressure system  and air flows of each states}
\fig{kaironaosi.eps}{width=0.7\hsize}{Circuit created}