\section{長時間動作が可能なウェアラブル椅子の開発}
\subsection{ウェアラブル椅子の開発}
コンセプトを基に開発したウェアラブル椅子を\figref{wearablechair.eps}に示す.
ウェアラブル椅子の重量は,5.7 kgであり,シリンダ非伸長時の幅・奥行・高さは,
425 × 360 × 420 mm,シリンダ伸長時の幅・奥行・高さは,425 × 360 × 620 mm である.
シリンダはコガネイ社製のCCDA50X200を使用している.シリンダの内径は50 mm,
ロッド径は,20 mm,ストロークが200 mmであり,質量は,1.68 kgである.
ロッドの先端はおねじが設けられており,先端に部品を付け替えることで,様々な場所での
作業に対応可能である.圧縮空気をためるタンクは,炭酸飲料用のペットボトルを使用している.
タンクは,炭酸飲料用の1 Lのペットボトルを圧力を供給するメインタンクに3本,空気回生機構に
使用する回生タンクに3本使用した.
一般的なペットボトルの耐圧が1.9~2.3 MPaである\cite{petto}.
圧縮空気を生成するのに必要なコンプレッサは,ウェアラブル
で移動可能な仕様にするために小型かつ軽量であることが求められる.コンプレッサは
SQUSE社 MP-2-2-Cを用いている.重さは0.194 kgであり,最大圧縮圧は,400 kPaである.
ウェアラブル椅子の構造はアルミフレームで構成されており,コの字は,関節構造となっている.
アルミフレームと座面と接触する内側とシャフトを止める外に各1個ずつ左右で,
計4個ベアリングを設けており,回転することが可能としている.このような関節構造
を設けることにより,移動の時は,エアシリンダと座面の角度を変更することが可能となり,
使用者がウェアラブル椅子を着用したまま移動の妨げにならないような構造となっている.
また,\figref{4.6.eps}に示すような,ロックピンを差し込むことでにシリンダと座面
とのなす角を90度に固定することが可能である.
\subsection{ウェアラブル椅子の装着}
\figref{3.7}(a)にウェアラブル椅子を装着した状態で立っている様子を示す.
ウェアラブル椅子と使用者は,Techouter社製のハーネスと,東洋物産工業社製の
バックル付きベルトで固定されている.\figref{3.7}(b)にロックピンを差し込み,
シリンダと座面の角度を90度に固定して座った状態を示す.\figref{3.7}(a)(b)に
示すように使用者にウェアラブル椅子を身に装着可能である.ロックピンでシリンダと座面
とのなす角を固定していない状態で歩行を行うと,使用者の脚にシリンダやタンクが当たり,
歩行の妨げになっていた.しかし,シリンダと座面の角度を90度に固定した状態で歩行すると,
シリンダが農作物などに引っかかる可能性がある.よって,シリンダと座面の角度を使用者の
歩行の妨げにならない角度で固定する必要があると考えられる.